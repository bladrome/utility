\documentclass[aspectratio=169]{beamer}
% other options: finnish, sectionpages

\usetheme{tut}

\usepackage[english]{babel}
\usepackage{listings}
\usepackage{csquotes}
\usepackage[style=authoryear,backend=biber]{biblatex}
\usepackage{filecontents}
\usepackage{blindtext}
\usepackage{pgfpages}

%\setbeameroption{show notes on second screen}% you'll need pgfpages for this one and a suitable pdf viewer i.e. dspdfviewer
%\setbeameroption{show notes}

\lstset{%
  basicstyle=\scriptsize\ttfamily,
  backgroundcolor=\color{TUTBlue}
}

\begin{filecontents}{bibliography.bib}
@book{hoenig1997,
 author = {Hoenig, Alan},
 title = {TEX Unbound: Latex and TEX Strategies for Fonts, Graphics and More},
 year = {1997},
 publisher = {Oxford University Press, Inc.}
} 
@misc{tantau2013,
  title = {The BEAMER class -- user guide for version~3.36},
  author = {T.~Tantau, J.~Wright, V.~Mileti{\'c}},
  year = {2015},
}
@online{tutgraphic,
  title = {Graphic guidelines},
  author = {{Tampere University of Technology}},
  note = {tutka > image > printed publications > graphic guidelines}
}
\end{filecontents}
\addbibresource{bibliography.bib}

% The color palettes can be changed
%   primary: lower half of titlepage, lower half of left color beam, default block environment
%   secondary: upper half of left color beam, example block body
%   tertiary: top header showing outline
%   quaternary: example block title
%   structure: structure elements such as list items
%\setbeamercolor{palette primary}{fg=white,bg=TUTsecPetrol}
%\setbeamercolor{palette secondary}{fg=white,bg=TUTsecOrange}
%\setbeamercolor{palette quaternary}{use=palette secondary,bg=palette secondary.bg!50!black}

\title{TUT Beamer Theme}
\subtitle{A theme for typesetting presentation slides using \LaTeX{}}
\author{Tuomas Välimäki}
\email{tuomas.s.valimaki@tut.fi}
\institute{Department of Automation Science and Engineering\\Tampere University of Technology}
\date{\today}

\begin{document}

\maketitle

\section*{Outline}
\begin{frame}{Outline}
	\tableofcontents
\end{frame}

\section{Introduction}
\subsection{Main features}
\begin{frame}{Main features}
  \begin{itemize}
    \item supports all aspect ratios
    \item vector backgrounds ``drawn'' using Ti\emph{k}Z
    \item customizable colors
    \item bilingual (finnish, english)
  \end{itemize}
  \bigskip
  Newest versions available under \url{https://github.com/tvalimaki/tut-beamer}
\end{frame}

\note{This is how a note page looks like if you use the beameroptions ``show notes'' or ``show notes on second screen''}

\subsection{Colors}
\begin{frame}{Colors, 1/2}
  \begin{columns}
  \begin{column}{0.475\textwidth}
    All colors introduced in TUT graphic guidelines are predefined.
  \end{column}
  \begin{column}{0.475\textwidth}
  Primary colors:\par
  \begin{tikzpicture}[%
      node distance=9em,
      every node/.append style={font=\scriptsize,
        minimum size=3ex}
  ]
    \node[rectangle,fill=TUTGreen,label=east:TUTGreen] (1) {};
    \node[rectangle,fill=TUTBlue, label=east:TUTBlue,right of=1] (2) {};
    \node[rectangle,fill=TUTGrey, label=east:TUTGrey,below of=1,node distance=4ex] (3) {};
  \end{tikzpicture}

  Secondary colors:\par
  \begin{tikzpicture}[%
      node distance=9em,
      every node/.append style={font=\scriptsize,
        minimum size=3ex}
  ]
    \node[rectangle,fill=TUTsecOrange,  label=east:TUTsecOrange]              (1) {};
    \node[rectangle,fill=TUTsecGreen,   label=east:TUTsecGreen,   right of=1] (2) {};
    \node[rectangle,fill=TUTsecPink,    label=east:TUTsecPink,    below of=1,node distance=4ex] (3) {};
    \node[rectangle,fill=TUTsecPetrol,  label=east:TUTsecPetrol,  right of=3] (4) {};
    \node[rectangle,fill=TUTsecPlum,    label=east:TUTsecPlum,    below of=3,node distance=4ex] (5) {};
    \node[rectangle,fill=TUTsecBlue,    label=east:TUTsecBlue,    right of=5] (6) {};
    \node[rectangle,fill=TUTsecRed,     label=east:TUTsecRed,     below of=5,node distance=4ex] (7) {};
    \node[rectangle,fill=TUTsecDarkblue,label=east:TUTsecDarkblue,right of=7] (8) {};
    \node[rectangle,fill=TUTsecDarkred, label=east:TUTsecDarkred, below of=7,node distance=4ex] (9) {};
  \end{tikzpicture}
  \end{column}
  \end{columns}
\end{frame}

\begin{frame}[containsverbatim]{Colors, 2/2}
  In addition to defining colors for single items, the color palettes of the theme can be changed.

  Try for example adding
  \begin{lstlisting}[%
    language={[LaTeX]TeX},
    texcsstyle=*\color{TUTsecOrange},
    moretexcs={setbeamercolor}
  ]
  \setbeamercolor{palette primary}{fg=white,bg=TUTsecPetrol}
  \setbeamercolor{palette secondary}{fg=white,bg=TUTsecOrange}
  \setbeamercolor{palette quaternary}{use=palette secondary,
                                      bg=palette secondary.bg!50!black}
  \end{lstlisting}
  to your preamble.
\end{frame}

\section{Example environments}
\subsection{Figures and equations}
\begin{frame}{Figures and equations}
  \begin{columns}[onlytextwidth]
    \begin{column}{0.5\textwidth}
        \centering
        \begin{figure}
        \includegraphicscopyright[width=\textwidth]{photo.jpg}{Image courtesy of \href{http://openphoto.net/gallery/image/view/5468}{openphoto.net}}
        \end{figure}
    \end{column}
    \begin{column}{0.4\textwidth}
    Here is some regular text in a column. And there is an equation
    \begin{displaymath}
      f(x)=ax^2+bx+c
    \end{displaymath}
    Here is some \alert{important} text.
    \end{column}
    \end{columns}
\end{frame}

\subsection{Lists}
\begin{frame}{List environments}
  \begin{columns}[onlytextwidth]
    \begin{column}{0.5\textwidth}
      This slide has a list\dots
      \blinditemize[3]
      \vspace*{5mm}
      descriptions\dots
      \blinddescription[2]
    \end{column}
    \begin{column}{0.5\textwidth}
      as well as some enumerations
      \blindenumerate[4]
    \end{column}
    \end{columns}
\end{frame}

\subsection{Blocks}
\begin{frame}{Block environments}
    \begin{exampleblock}{Example}
        This is an example
    \end{exampleblock}
    
    \begin{alertblock}{Note}
        This is important
    \end{alertblock}

    \begin{theorem}[Pythagoras] 
        $ a^2 + b^2 = c^2$
    \end{theorem}
\end{frame}

\section{Further reading}
\begin{frame}{Further reading}
  \nocite{*}
  \printbibliography[heading=none]
\end{frame}

\end{document}
